\documentstyle{letter}

\input head   %VENGONO CARICATE LE DEFINIZIONI CHE IN lettera1.tex
              %COMPAIONO NEL PREAMBOLO

%% \address{ESRF -- European Synchrotron Radiation Facility\\ Computer
%% Service Division\\ 6, rue Jules Horowitz\\ 38043 Grenoble, France}
\address{Dr. Giuseppe Mazza \\via Tuscolana 687/16B \\00174 Roma, Italy}
%% \address{Dr. Giuseppe Mazza \\20 avenue Alsace--Loraine  \\38000
%% Grenoble, France}

\signature{Dr. Giuseppe Mazza}

\begin{document}    %INIZIO DOCUMENTO

%% \date               %GENERAZIONE DELLA DATA ODIERNA

%% \begin{letter}{Dr. Felice Destinatario\\ Dipartimento di Fisica\\ ROMA} 
%% ----------------------------------------------------------------
%%
%% OLD
%%
%% \begin{letter}{Professor Peter R Hobson\\ School of Engineering and Design\\ Brunel University\\ Uxbridge UB8 3PH}
%% ----------------------------------------------------------------
%% \begin{letter}{Information Technology Department \\Columbia University \\ Morningside, New York}
%% \begin{letter}{DESY -- Deutsches Elektronen-Synchrotron \\Human Resources Department \\Notkestrae 85 j \\22607 Hamburg, Germany}
%%  \begin{letter}{New York University \\Washington Square Park \\New York, New York (US)}
%% \begin{letter}{United Nations Office for Project Services (UNOPS) \\United Nations Department of Field Support, Information and Communications Technology Division (UN DFS--ICTD)}
%% \begin{letter}{United Nations Office for Project Services (UNOPS)}
%% \begin{letter}{The University of Melbourne \\Victoria 3010 \\Australia}
\begin{letter}{EMBL-EBI \\Personnel \\Wellcome Trust Genome Campus \\Hinxton, CB10 1SD}
%% \begin{letter}{SWITCH \\PO Box, CH-8021 \\Zurich, Switzerland}
%% \begin{letter}{NIHR Biomedical Research Centre \\for Mental Health Nucleus at IOP \\Kings College London}
%% \begin{letter}{EGI -- Stichting European Grid Initiative \\105
%% Science Park \\1098 XG Amsterdam \\The Netherlands}
%% \begin{letter}{ESRF -- European Synchrotron Radiation Facility\\ Computer
%% Service Division\\ 6, rue Jules Horowitz\\ 38043 Grenoble, France}
%% \begin{letter}{EMBL Personnel \\Postfach 10.2209 \\69012 Heidelberg \\Germany}
%% \begin{letter}{United Nations Population Fund (UNFPA) \\Coordination and Support Services \\Copenhagen -- Denmark} 
%% \begin{letter}{}


\open{Dear Madam/Sir,}
%% \open{Dear Louise,}


My name is Giuseppe Mazza. I used to work as Linux System Administrator (``Ingenieur'')
in the Computer Service Division for ESRF (European Synchrotron Radiation Facility) 
in Grenoble, France. 
%% e-Science Systems Manager
%% in the Department of Physics at Queen Mary, University of London.

%% I have been responsible for the hardware and software management
%% of the Queen Mary's High Throughput Cluster (almost 500 servers)
%% for about three years.
%% I used to be a linux user, then a windows user and now a mac one

I would like to apply
to the EMBL-EBI Hinxton, United Kingdom,
for the position whose job refecence is 
ref. no. W/10/029/EBI (``User Support Officer'').
% I would like to apply
% to the EMBL-EBI Hinxton, United Kingdom,
% for the position whose job refecence is 
% ref. no. W/10/016/EBI (``Ensembl Cloud Computing Specialist'').


Here are names and addresses of my referees:
\begin{itemize}
\item Bruno Lebayle, lebayle@esrf.fr 
  (Group Head Computing Services,
  ESRF - European Synchrotron Radiation Facility
  Grenoble, France).
\item Professor Roberto Petronzio, petronzio@roma2.infn.it
  (President of INFN,  Italian National Institute for Nuclear Physics
  Rome, Italy).
\end{itemize}

% unix sys admin, near Cambridge
% to the 
% EBI Systems Group
% within
% EMBL
% for the position
% ``IT System Engineer, Linux''.
% (ref: 08/73ARE).


%% ----------------------------------------------------------------
% I worked in a large scale enterprise environment (almost 500 servers)
% at Queen Mary, University of London.
% I used to be responsible for the day--to--day management
% of the HTC Cluster hardware and software.

% My duties included:
% \begin{itemize}
% \item install the Grid middleware.
% \item maintain local package repositories: Scientific Linux (RedHat Linux from Cern) and  the Grid middleware. 
% \item day to day administration of Linux server systems.
% \item monitoring and troubleshooting of the local batch--system. 
% \item act as a point of contact for technical systems support, troubleshooting and advice to staff and postgraduates.
% \item represent Queen Mary College -- in particular the Physics
%   Department-- to the GridPP Meetings around UK and 
% in Europe (Cern Geneve, Edimburgh, Dublin, ...)
% \item I took part at the QM cluster upgrade in 2007: the number of nodes grew from about 170 to about 500.
% \end{itemize}


% I managed Linux clusters at ESRF as well:
% \begin{itemize}
% \item an HTC cluster with a Condor job scheduler and a monitoring based on
% Ganglia.
% \item an other cluster running scientific software (Matlab, Mathematica, etc.) 
% \end{itemize}

% Morever I realized the following project at the ESRF: I built and managed 
% a five host xen pool (Citrix XenServer). 
%% ----------------------------------------------------------------

% unix sys admin, near Cambridge
% to the 
% EBI Systems Group
% within
% EMBL
% for the position
% ``IT System Engineer, Linux''.
% (ref: 08/73ARE).

% per conoscere le istruzioni generalmente usate per generare un documento
% stile lettera occorre processare con \LaTeX\ il file {\bf makeletter.tex}. \\
% Per vedere l'effetto che i nuovi comandi hanno sull'output, basta
% confrontare il presente testo di input con l'output che esso produce, dopo
% essere stato processato dal \LaTeX. \\

% Ricordarsi di caricare sempre il file {\em head.tex} con il comando
% $\backslash input$ posto nel preambolo. Tale file contiene le definizioni
% della data, di {\em open} e di {\em close} che servono per comporre la 
% lettera e che nel file {\em lettera1.tex} compaiono invece nella regione
% del preambolo.\\

% Per ottenere l'intestazione e la data a sinistra e il destinatario e il
% mittente a destra, come nella presente lettera, si usano le definizioni
% OPEN e CLOSE in sostituzione dei comandi \LaTeX\ OPENING  e CLOSING.

% \close{Cordiali saluti.}
\close{Best Regards,}
%Altrimenti chiudere con \close{\ }

\end{letter}

\end{document}
