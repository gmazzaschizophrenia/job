\begin{rubric}{Professional experience}

\entry*[2010--now]
Systems Programmer  in the Computing Support Group (CSG), Department
of Computing, Imperial College London. Duration: 10.05.2010 -- open--ended.
%% line managers: Anne O'Neill, Duncan White

\entry*[2008--10]
A 1-and-half year contract at ESRF (European Synchrotron Radiation Facility)
in Grenoble, France. Duration: 04.08.2008 -- 03.02.2010.
Job: Linux System Administrator (``Ing\'enieur'').

\entry*[2005--8]
A 2--year PPARC (Particle Physics and Astronomy Research Council) Grant
at Queen Mary, University of London.
%% line manager Dr A.J.Martin,
%% supervisor Prof S.L.Lloyd,
Duration: 01.07.05 -- 30.06.07.
Job: e--Science Systems Manager -- Research Associate.
A contract extension (1 year): 01.07.07 -- 30.06.08.

% \entry*[]
% % 1st contract extension (3 months): 01.07.07 -- 30.09.07
% A contract extension (1 year): 01.07.07 -- 30.06.08.

\entry*[2004--5]
A 2--year INFN (Italian National Institute for Nuclear
Physics) fellowship at University of Rome ``Tor Vergata'',
tutor Prof R. Petronzio. Duration: 01.12.03 -- 30.11.05.
% Research activity:
% ``Sviluppo  di  una libreria di tipo MPI %% (Message Passing Interface)
% per Cluster di PC con schede di comunicazione proprietarie''
(R.\ Ammendola et al., APENet: LQCD clusters \`a la APE, 
% Proceedings of the XXII International Symposium on Lattice Field
% Theory (LATTICE $2004$) on June $21$ -- $26$, $2004$ in Batavia,
% Illinois, USA
http://arxiv.org/pdf/hep-lat/0409071
).

\entry*[2003]
Fixed term contract at University of Rome ``Tor Vergata'',
tutor Prof R. Petronzio. Duration 01.03.03 -- 31.06.03.
Field: 
``Numerical simulation of spin glass systems using parallel algorithms''.
% ``Sviluppo e calcolo numerico di codice di spin glass ottimizzato per
% algoritmi parallelizzabili''(``Numerical simulation of spin glass systems using
% parallel algorithms'').

\end{rubric}
