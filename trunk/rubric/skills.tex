\begin{rubric}{Computer skills}

\subrubric{System administration}

\entry*[9 years]
%% Installation of Fedora (Red Hat), SuSE, Debian,
%% Free BSD, Solaris, Windows Xp
Scientific Linux (Fedora, Red Hat, Ubuntu) (10 years), MacOS X,
Windows Xp.
%% , Solaris9.
Unix--to--Windows interoperability: VMware, Cygwin.
%% --------

\entry*
I worked in a large scale enterprise environment (almost 500 servers)
at Queen Mary, University of London.
% I used to be responsible for the day--to--day management
% of the HTC Cluster hardware and software.


%% Networking: iptables, dhcp, bind, nfs.
%% Networking: 
I used iptables, dhcp, dns, nfs. %% apache, samba. 
I got some experience with MySQL and Apache
too.
%% Clustering: torque/maui, SGE, condor.
%% Grid sw: lcg2.7, glite3.0, glite3.1
%% Storage: 3ware RAID systems, Areca RAID systems.
%% --------

\entry*
%% I realized the following projects at the ESRF: 
I built and managed a five host xen pool (Citrix XenServer) at ESRF.
%% I installed various VMs: Windows2003, Vista, Xp, RHLE5.
%% I setup a 4 node HA cluster which hosts two plone implemented websites.
% I managed Linux clusters running scientific software (Matlab, Mathematica, etc.)
% and a software license Solaris9 server.
I got some knowledge of the following products: Tivoli, EFFIP, 
NetApp Filer, TiNa, Solaris10.

\entry*
I started using Ubuntu Linux. I have been using Perl since my first
day at Imperial.

%% Storage: Adaptec RAID cards on sun x4150 systems, NetApp, TINA.
%% --------

% \entry*
% ``Idoneit\`a'' to the public competition at University of Rome ``RomaTre'':
% examination for a D~category non--fixed term contract,
% economic position 1
% -- technical, techno--scientific area  and data management --
% for the requirements of the Department of Physics
% ($3^o$ classified, voting 23,5/30),
% published since 10.12.2003 like official notice in
% ``GU n. 99, 19.12.2003, della Repubblica Italiana''
%% ----------------------------------------------------------------

\subrubric{Programming}

\entry*[4 years]
C/C++ languages and programming in Linux environment (emacs, vi, gcc, make 
\& Makefile, diff, gdb, git (svn, cvs), awk, grep, find, bash, 
%% python, 
perl, 
\LaTeX).
%% : experience during work at APE Lab of INFN.

%% \entry*
%% Open source libraries:
%% focus in parallel computing using
%% MPI Standard (LAM/MPI and MPICH libraries); Fast Fourier Trasform (FFTW) library

%% \entry*
%% Web development: xhtml, css, php (quanta)
%% ----------------------------------------------------------------

\end{rubric}
